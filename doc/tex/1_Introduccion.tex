\capitulo{1}{Introducción}

En el proyecto ``Generación de Rutas Turísticas Personalizadas'', realizado por Íñigo Vázquez Gómez y Roberto Villuela Uzquiza \cite{grpV1} se presentó una herramienta capaz de generar rutas en una ciudad introduciendo simplemente un tiempo, un origen, un destino y unas preferencias en cuanto a intereses.
Esta aplicación fue ampliada un año después por el alumno Alejandro Cuevas Álvarez, bajo el título ``Ampliación de la Aplicación para la Generación de Rutas Turísticas Personalizadas''\cite{grpV2}.

En este documento se van a presentar las ampliaciones realizadas a ese último proyecto, más concretamente al \textit{BackEnd}. Se ha querido mantener la funcionalidad del proyecto original en su totalidad, aportando a este nuevas formas de generar rutas turísticas teniendo en cuenta detalles como los horarios de los establecimientos o el tiempo que el visitante va a pasar en cada punto. De esta forma logramos obtener resultados que se adapten mejor a cada situación, mejorando enormemente la experiencia del usuario.

Para la realización de este proyecto ha sido necesario involucrarse en diferentes áreas dentro de la Informática, como pueden ser algoritmia, bases de datos y sistemas distribuidos.

Se han mantenido las decisiones tomadas en el proyecto original, por lo que se ha utilizado el lenguaje de programación orientado a objetos \textit{Java}.
Se dispone de una base de datos \textit{PostgreSQL} \cite{wiki:postgresql}, en la que se encuentra almacenada información como puntos de interés, sus características, relaciones entre puntos, usuarios, rutas...
 