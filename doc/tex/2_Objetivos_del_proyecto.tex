\capitulo{2}{Objetivos del proyecto}

Los objetivos que se muestran a continuación están basados en un punto principal, que es la mejora de la aplicación ya existente, añadiendo nuevas funcionalidades que la hagan más atractiva para el usuario.

Se van a llevar a cabo las siguientes mejoras:
\begin{itemize}
\tightlist
\item Actualizar la base de datos con información referente a horarios de apertura y cierre de los distintos puntos de interés.
\item Actualizar las etiquetas de los diferentes puntos de interés para poder buscar por filtros personalizados.
\item Implementar un algoritmo que tenga en cuenta los horarios de apertura y cierre de los distintos puntos de interés, para añadir realismo a la aplicación y generar rutas más eficientes, ya que el modelo anterior, al no tener en cuenta estos datos, podía añadir a la ruta un punto de interés "no visitable".
\item Implementar la funcionalidad de no repetir puntos de interés en caso de estar planificando la ruta en una ciudad ya visitada.
\end{itemize}

Los objetivos personales son:
\begin{itemize}
\tightlist
\item Aprender a utilizar las herramientas PostgreSQL, PostGIS y GlassFish para tener una base sólida en el campo de creación y uso de servidores.
\item Aprender a utilizar el sistema de composición de textos LaTeX, así como la herramienta de desarrollo Texmaker.
\item Aprender a trabajar en un "equipo de desarrollo" y mejorar en el mismo para tener una experiencia más parecida a la que podemos encontrarnos a la hora de realizar un proyecto "real".
\end{itemize}