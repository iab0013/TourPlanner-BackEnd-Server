\capitulo{3}{Conceptos teóricos}

%En aquellos proyectos que necesiten para su comprensión y desarrollo de unos conceptos teóricos de una determinada materia o de un determinado dominio de conocimiento, debe existir un apartado que sintetice dichos conceptos.
%
%Algunos conceptos teóricos de \LaTeX \footnote{Créditos a los proyectos de Álvaro López Cantero: Configurador de Presupuestos y Roberto Izquierdo Amo: PLQuiz}.
%
%\section{Secciones}
%
%Las secciones se incluyen con el comando section.
%
%\subsection{Subsecciones}
%
%Además de secciones tenemos subsecciones.
%
%\subsubsection{Subsubsecciones}
%
%Y subsecciones. 
%
%
%\section{Referencias}
%
%Las referencias se incluyen en el texto usando cite \cite{wiki:latex}. Para citar webs, artículos o libros \cite{koza92}.
%
%
%\section{Imágenes}
%
%Se pueden incluir imágenes con los comandos standard de \LaTeX, pero esta plantilla dispone de comandos propios como por ejemplo el siguiente:
%
%\imagen{escudoInfor}{Autómata para una expresión vacía}
%
%
%
%\section{Listas de items}
%
%Existen tres posibilidades:
%
%\begin{itemize}
%	\item primer item.
%	\item segundo item.
%\end{itemize}
%
%\begin{enumerate}
%	\item primer item.
%	\item segundo item.
%\end{enumerate}
%
%\begin{description}
%	\item[Primer item] más información sobre el primer item.
%	\item[Segundo item] más información sobre el segundo item.
%\end{description}
%	
%\begin{itemize}
%\item 
%\end{itemize}
%
%\section{Tablas}
%
%Igualmente se pueden usar los comandos específicos de \LaTeX o bien usar alguno de los comandos de la plantilla.
%
%\tablaSmall{Herramientas y tecnologías utilizadas en cada parte del proyecto}{l c c c c}{herramientasportipodeuso}
%{ \multicolumn{1}{l}{Herramientas} & App AngularJS & API REST & BD & Memoria \\}{ 
%HTML5 & X & & &\\
%CSS3 & X & & &\\
%BOOTSTRAP & X & & &\\
%JavaScript & X & & &\\
%AngularJS & X & & &\\
%Bower & X & & &\\
%PHP & & X & &\\
%Karma + Jasmine & X & & &\\
%Slim framework & & X & &\\
%Idiorm & & X & &\\
%Composer & & X & &\\
%JSON & X & X & &\\
%PhpStorm & X & X & &\\
%MySQL & & & X &\\
%PhpMyAdmin & & & X &\\
%Git + BitBucket & X & X & X & X\\
%Mik\TeX{} & & & & X\\
%\TeX{}Maker & & & & X\\
%Astah & & & & X\\
%Balsamiq Mockups & X & & &\\
%VersionOne & X & X & X & X\\
%} 

En este capítulo se describirán, como ya anticipa el título, diversos aspectos relacionados con cuestiones teóricas, que permitirán al lector una correcta comprensión del trabajo.

\section{Sistema de información geográfica}
%Sección datos GIS
Un sistema de información geográfica (GIS) \cite{wiki:gis} es un conjunto de herramientas que integra y relaciona diversos componentes que permiten la organización, almacenamiento, manipulación, análisis y modelización de grandes cantidades de datos procedentes del mundo real que están vinculados a una referencia espacial, facilitando la incorporación de aspectos sociales-culturales, económicos y ambientales que conducen a la toma de decisiones de una manera más eficaz.

Funcionalidad de los GIS:
\begin{enumerate}
\item Almacenar información, que se puede obtener por diferentes métodos, como pueden ser: GPS, fotografía aérea, imágenes satélite, bases de datos \ldots
\item Visualizar datos almacenados.
\item Hacer consulta sobre los datos seleccionados. De esta forma podemos presentar la información de una forma mucho más amigable para el usuario, mediante mapas, gráficos, tablas \ldots
\item Hacer análisis para generar nuevas capas de información útil.
\end{enumerate}

Los tipos de datos geográficos más utilizados actualmente por las GIS son:
\begin{itemize}
\item Datos Vectoriales, utilizados para representar fenómenos discretos, describiendo objetos geográficos a partir de vectores definidos por pares de coordenadas.
\item Datos Raster, utilizdos para representar fenómenos no discretos. Divide el área a representar en una retícula de celdas y atribuye a cada una de estas un valor numérico para representar su valor temático. Cabe destacar que el origen siempre esta en la esquina superior izquierda de la retícula.
\end{itemize}


\section{GPS}
Sección datos GPS