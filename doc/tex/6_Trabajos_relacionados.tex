\capitulo{6}{Trabajos relacionados}

\section{TripAdvisor (IOS/Android)}
Sin duda una de las aplicaciones para planificar viajes más populares. Nos permite seleccionar una ciudad y descubrir puntos de interés de diferentes categorías, como pueden ser:
\begin{itemize}
\item Sitios más fotografiados
\item Lugares más emblematicos
\item Restaurantes según su precio
\item Establecimientos especializados
\item Hostales y hoteles
\end{itemize}

Permite subir fotos y vídeos a la aplicación para que el resto de usuarios puedan verlos y ofrece diferente información sobre el lugar recopilada de periódicos como el ABC, guías de viajes etc.

Una de las opciones más interesantes que presenta esta aplicación es la posibilidad de acceder a sus foros, donde los usuarios realizan preguntas sobre qué lugares recomiendan en la ciudad a la que van a viajar, que comer etc. De esta forma se logra una comunidad conectada y da la sensación de trato personal.

Cabe destacar que al instalar la aplicación nos da la opción de registrarnos utilizando una cuenta de \textit{Gmail} o \textit{Facebook}, pero no es obligatorio. En caso de no registrarnos:
\begin{itemize}
\item No se nos dejará acceder al buzón, donde recibimos notificaciones de nuestro muro de viajes y chats privados con otros usuarios.
\item No se nos dejara acceder a la sección de viajes.
\end{itemize} 

El uso de la aplicación es extremadamente sencillo, basta con dar un nombre a un viaje, que en realidad es un repositorio donde podremos guardar lugares para visitar más adelante.

No genera rutas automáticamente, sino que funciona más bien como planificador de viajes.

\section{Nativoo Guía de Viajes}










