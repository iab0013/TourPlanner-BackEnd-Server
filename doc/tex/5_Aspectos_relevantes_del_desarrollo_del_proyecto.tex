\capitulo{5}{Aspectos relevantes del desarrollo del proyecto}

%Este apartado pretende recoger los aspectos más interesantes del desarrollo del proyecto, comentados por los autores del mismo.
%Debe incluir desde la exposición del ciclo de vida utilizado, hasta los detalles de mayor relevancia de las fases de análisis, diseño e implementación.
%Se busca que no sea una mera operación de copiar y pegar diagramas y extractos del código fuente, sino que realmente se justifiquen los caminos de solución que se han tomado, especialmente aquellos que no sean triviales.
%Puede ser el lugar más adecuado para documentar los aspectos más interesantes del diseño y de la implementación, con un mayor hincapié en aspectos tales como el tipo de arquitectura elegido, los índices de las tablas de la base de datos, normalización y desnormalización, distribución en ficheros3, reglas de negocio dentro de las bases de datos (EDVHV GH GDWRV DFWLYDV), aspectos de desarrollo relacionados con el WWW...
%Este apartado, debe convertirse en el resumen de la experiencia práctica del proyecto, y por sí mismo justifica que la memoria se convierta en un documento útil, fuente de referencia para los autores, los tutores y futuros alumnos.
\section{Algoritmos para el cálculo de rutas}
Una de las decisiones más importantes del proyecto es la de elegir que algoritmo implementar, ya que es el pilar central de la ampliación que se quiere realizar.

El problema se trata de una variante del algoritmo base, que es el \textit{problema de la orientación} (\textit{OP}), detallado en la sección \textit{Conceptos teóricos}, concretamente la variante \textit{problema de la orientación con ventanas de tiempo} (\textit{OPTW}).

Las bases del algoritmo son sencillas: se quiere visitar una ciudad y se han de seleccionar que puntos de interés (ordenados) crean la ruta más óptima. No suele darse en caso en que se seleccionen todos los puntos disponibles, pero esto es por las restricciones de tiempo.

La diferencia principal con el algoritmo base es que el tiempo de viaje entre puntos ya no es el único tiempo a tener en cuenta. A cada punto de interés se le asigna una ventana de tiempo en la que será obligatorio llegar para que el punto sea visitable. Se puede pensar que la solución a este problema es tan sencilla como añadir una condición al proceso de selección, pero la verdadera complicación es cómo reducir lo máximo posible los tiempos de cómputo.

La decisión de que algoritmo de los disponibles implementar se basó en su extensibilidad, es decir, debido a que OPTW no es la única variación del algoritmo base, se eligió una solución que permitiese, con no demasiados cambios, resolver alguna de dichas variaciones en futuros proyectos. Dicho algoritmo el el basado en \textit{colonia de hormigas}.

Una de las desventajas que se encontraron con ese algoritmo fue la falta de tablas en las que se mostrasen comparaciones de los tiempos de cómputo, aunque ya se adelantaba que eran altos.
Es por eso que se decidió implementar un segundo algoritmo, a poder ser basado en una resolución lo más diferente posible, para así poder comparar como de eficiente era cada uno. Este segundo algoritmo está basado en \textit{búsquedas iterativas}.

Los dos algoritmos solución elegidos fueron:
\begin{itemize}
\tightlist
\item Iterated local search for OPTW
\item Ant Colony System
\end{itemize}
Las implementaciones de las dos soluciones se encuentran explicadas detalladamente en el \textit{Anexo IV: 2.Algoritmos para el cálculo de rutas}.

\section{Obtención y tratamiento de datos}

\subsection{Obtención de horarios}
Se evaluaron dos posibles fuentes de datos:
\begin{itemize}
\item Google Maps
\item OpenStreetMaps
\end{itemize}

Lo primero que se investigó fue el formato de los datos en cada una de las opciones, centrándonos sobre todo en los horarios de los establecimientos ya que eran los nuevos datos a implementar.

Por un lado \textit{Google Maps} es uno de los servicios más utilizados cuando se requieren datos de geolocalización. Cuenta con una API\cite{wiki:googlemapsapi} relativamente sencilla de usar, ofrece unos datos en un único formato y que (normalmente) han sido verificados, lo cual la hace una elección muy llamativa.

Cada punto cuenta con 3 categorías y cada una contiene diferentes campos. Dichas categorías y datos son:

\textbf{Basic}
\begin{itemize}
\tightlist
\item address\_component
\item adr\_address
\item formatted\_address
\item geometry
\item icon
\item name
\item permanently\_closed
\item photo
\item place\_id
\item plus\_code
\item type
\item url
\item utc\_offset
\item vicinity
\end{itemize}

\textbf{Contact}
\begin{itemize}
\tightlist
\item formatted\_phone\_number
\item international\_phone\_number
\item opening\_hours
\item website
\end{itemize}

\textbf{Atmosphere}
\begin{itemize}
\tightlist
\item price\_level
\item rating
\item review
\item user\_rattings\_total
\end{itemize}

Este servicio no solo ofrecía los datos necesarios, sino que aportaba otros extra que podían ser utilizados para hacer la aplicación más realista, permitiendo por ejemplo descartar ciertos puntos de interés en función del precio.

Al ir a implementar la API se encontró un obstáculo insalvable, las licencias de Google Maps.
Si bien es cierto que la restricción del número de elementos por consulta (100 elementos) ha existido desde el nacimiento de esta API, a partir del 16 de Julio de 2018 es necesario aportar los datos de facturación en la cuenta con la que se utiliza dicho servicio, ya que en esta fecha se estableció un coste por el uso de los datos.

Los algoritmos utilizados necesitan una gran cantidad de datos, ya que para disminuir el tiempo de cómputo se utiliza una \textit{matriz de distancias}, en la que se cargan los datos de localización de cada punto para facilitar la obtención del espacio entre estos.

Aunque era posible solventar la restricción de peticiones disminuyendo considerablemente los datos disponibles para los algoritmos, no había opción con el pago por el servicio, por lo que se tuvo que descartar la opción de utilizar Google Maps.

La segunda opción disponible era utilizar el proyecto colaborativo \textit{OpenStreetMaps}\cite{wiki:openstreetmaps}, que contaba con la ventaja de tener los datos abiertos al publico, ya que es la comunidad quien se encarga de aportar dichos datos. Para ello el único requisito es estar registrado.

Este último punto plantea otro inconveniente, dado que cualquiera puede añadir datos nuevos (o corregir los antiguos), era posible encontrarnos con ciertas incongruencias en los datos, sobre todo en los referentes a horarios de apertura y cierre de los establecimientos.

Otros inconvenientes de utilizar OpenStreetMaps eran:
\begin{itemize}
\tightlist
\item Falta de una API
\item Documentación oficial escasa del uso de sus herramientas
\item Necesidad de tener los datos previamente volcados en nuestro servidor para poder utilizarlos
\item Necesidad de una base de datos GIS
\end{itemize}

Dado que no había otra alternativa, la elección estaba tomada sin importar los inconvenientes. 

Cabe destacar el gran trabajo que hicieron los anteriores contribuyentes de este proyecto: Íñigo Vázquez Gómez, Roberto Villuela Uzquiza y Alejandro Cuevas Álvarez. Su trabajo facilitó mucho la comprensión y el uso de esta herramienta

\subsection{Volcado de datos, Osmosis vs Osm2pgsql}
Se investigó que herramientas existían para el volcado de los datos presentes en los ficheros descargados de \textit{OpenStreetMaps} a nuestra base de datos geoespacial. Se valoraron dos alternativas:
\begin{itemize}
\item Osmosis
\item Osm2pgsql
\end{itemize}

Una de las ventajas de  Osmosis es que los datos se vuelcan de una forma muy particular, creando relaciones de las que podíamos beneficiarnos como ya habían hecho en años anteriores el resto de contribuyentes de esta aplicación.

Una vez volcados los datos se reparó en que no solo no se había importado la etiqueta opening hours, sino que no se había importado ninguna etiqueta que no fuese de las principales, entendiendo por principales:
\begin{itemize}
\item ID
\item Nombre
\item Localización
\item Tipo de nodo/vía
\item Relaciones entre nodos
\end{itemize}

En vista del resultado se pasó a utilizar la herramienta \textit{Osm2pgsql} ya que consta de un archivo de estilo, llamado \textit{default style}, en el que se pueden indicar que etiquetas de todas las existentes en OpenStreetMaps queremos importar.
El volcado se realizó con éxito, pero el formato de las tablas creadas en \textit{PostgreSQL} no eran adecuadas para el uso que se les quería dar.

Finalmente, se decidió que, debido a que cada herramienta aportaba utilidades imprescindibles, se utilizarían ambas herramientas, beneficiándonos así de sus puntos fuertes y solventando las debilidades.
La solución final se encuentra explicada detalladamente en el \textit{Anexo RELLENAR ANEXO}.

Cabe destacar que en caso de querer volcar una gran cantidad de datos, como por ejemplo, de toda España, existe la posibilidad de modificar la cantidad de memoria de nuestro PC que permitimos utilizar a Osm2pgsql. Si dicha cantidad no es suficiente, a mitad del volcado se quedará sin espacio y se nos notificará con un error que no se han podido volcar los datos. Por el contrario, si ofrecemos demasiada memoria nuestro ordenador dejará de responder y tendremos que forzar un apagado del sistema manteniendo pulsado el botón de encendido.

Se recomienda que, en caso de necesitar utilizar ficheros tan pesados, es mejor obtener cortes de dicho fichero, por ejemplo, por comunidades autónomas o por provincias y realizar el volcado de uno en uno, para así evitar errores como los mencionados.

\subsection{Tratamiento de horarios}
Tras el volcado de datos se pasó a comprobar la calidad de la información.

La etiqueta \textit{opening\_hours} cuenta con una sintaxis bastante compleja, que bien utilizada es capaz de aportar una gran información. Un ejemplo de dicha sintaxis es el siguiente:
\imagen{opHoursExample}{Ejemplo de sintaxis de la etiqueta opening hours \cite{wiki:opHours}}

Como se ha comentado con anterioridad, OpenStreetMaps permite a sus usuarios subir sus propios datos, lo que causa que algunos de ellos contengan una sintaxis errónea.

Después de comentar con el tutor del proyecto las posibles soluciones, se llegó a dos ideas:
\begin{itemize}
\item Gestión de los datos desde la base de datos
\item Gestión de los datos desde los algoritmos
\end{itemize}

Las comprobaciones de los datos mostraban que una gran cantidad de puntos de interés no contaban con un horario, pero no todos los usuarios habían optado por no introducir ningún valor, ciertos establecimientos tenían en su horario ``", lo cual no era detectado ni como nulo ni como espacio. Esto dio varios dolores de cabeza a la hora de probar los algoritmos hasta que se descubrió que esta era la causa.

Se decidió por tanto practicar las dos soluciones a la par, nada más introducir los datos se usaría un script SQL que buscaría aquellos que contuviesen tanto nulo como ``" y sustituiría esos valores por unos por defecto.

Por otro lado, se estudiaron todos los posibles tipos de sintaxis y se creó un método dentro del algoritmo que extrajese los horarios de apertura y cierre a partir de un String obtenido de la base de datos.

Estas soluciones se encuentran explicadas mas detalladamente en el \textit{Anexo RELLENAR ANEXO}

\section{Conocimientos adquiridos en la carrera}
\subsection{Conocimientos sobre inteligencia artificial}
Ha sido de gran ayuda para este proyecto haber cursado durante la carrera la asignatura \textit{Computación Neuronal y Evolutiva}, ya que tuvimos nuestro primer acercamiento a este tipo de algoritmos mediante un desafio Hash Code, así como las asignaturas \textit{Sistemas Inteligentes}, \textit{Algoritmia}, la asignatura \textit{Advanced Logical and Functional Programming} cursada en la \textit{West University of Timisoara} etc.
Gracias a esto la comprensión de artículos referentes a la resolución e implementación del problema de la orientación ha sido mucho mas llevadera.

\subsection{Conocimientos sobre bases de datos}
De vital importancia ya que ha sido necesario crear una base de datos GIS y modificarla mediante SQL para que fuese más eficiente.
Gracias a haber cursado las asignaturas de \textit{Bases de Datos y Aplicaciones de Bases de Datos} se ha partido de una posición más avanzada y ha sido posible tanto la creación de la base de datos como el manejo del álgebra relacional.

\subsection{Trabajo de investigación}
No se ha impartido en ninguna asignatura en concreto, pero ha estado presente en cada uno de los trabajos que se han desarrollado en el grado.

La rapidez con la que se buscaban y comprendían nuevos conocimientos, se verificaban las fuentes de los datos y se solventaban los diversos problemas y errores que han aparecido en el desarrollo de este proyecto ha demostrado que esta capacidad es una de las más importantes que se han adquirido durante la carrera. 
