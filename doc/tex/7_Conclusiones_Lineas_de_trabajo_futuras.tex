\capitulo{7}{Conclusiones y Líneas de trabajo futuras}
En este apartado se tratarán las diferentes conclusiones que han surgido durante la realización de este proyecto así como posibles lineas que seguir en un futuro para mejorar la funcionalidad de la aplicación.

\section{Conclusiones}
Durante la realización de este proyecto se ha aprendido a utilizar repositorios como OpenStreetMaps para obtener datos geográficos, a crear y utilizar bases de datos geoespaciales, a trabajar con proyectos Java bajo la herramienta de gestión Maven, a implementar y comprender diferentes algoritmos para resolver el problema de la orientación, a trabajar con el servidor de aplicaciones Glassfish y a tratar con los diversos problemas que genera el trabajar en la ampliación de un proyecto anterior.

Como valoración personal, con este proyecto se han conseguido reforzar diferentes conocimientos adquiridos a lo largo de toda la carrera, muchos de los cuales no se habían puesto en práctica.

\section{Líneas de trabajo futuras}
En este apartado se comentarán las posibles mejoras que se podrían aplicar a este proyecto:
\begin{itemize}
\item Ampliar la generación de rutas, haciendo al algoritmo dependiente del tiempo, es decir, que la velocidad de ruta varíe en función de la hora del día (Time Dependent Orienteering Problem). Los algoritmos implementados en este proyecto están preparados para dicha ampliación.
\item Crear una página web que iguale la funcionalidad de la aplicación.
\item Conectar la aplicación con las redes sociales, para que permita compartir y publicar las rutas generadas.
\item Conectar la aplicación con el GPS del dispositivo, para que pueda guiarnos hacia el siguiente punto de interés.
\item Actualizar el cliente a una versión de Android más actual.
\item Crear una versión del cliente para otros sistema móviles como pueden ser IOS o Windows Phone.
\item Incluir realidad aumentada en la aplicación, para poder ver los diferentes puntos de interés sin necesidad de hacer la ruta personalmente.
\item Conectar la aplicación a diferentes funciones del dispositivo para, en función de la distancia, el tiempo recorrido y los pasos dados, proveer al usuario información sobre su salud física.
\item Un sistema de amigos con el que se puedan ver las últimas rutas de estos.
\end{itemize}