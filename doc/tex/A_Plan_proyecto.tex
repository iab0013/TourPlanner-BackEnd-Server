\apendice{Plan de Proyecto Software}

\section{Introducción}
A la hora de realizar un proyecto es imprescindible dedicar tiempo a la primera fase del mismo, la planificación. Es una fase importante ya que de ella depende el alcanzar los objetivos propuestos.

En este anexo se detallarán las tareas realizadas así como el tiempo dedicado a cada una de estas con el fin de acortar el tiempo de producción del proyecto.

Por otro lado se tendrá en cuenta un punto de vista más económico para valorar la posibilidad de llevar el producto final a un ámbito comercial.

Nos centraremos en los siguientes puntos:
\begin{itemize}
\item Planificación temporal
\item Estudio de viabilidad
\begin{itemize}
\item Viabilidad económica
\item Viabilidad legal
\end{itemize}
\end{itemize}
\section{Planificación temporal}
Como se mencionó en la Memoria, se ha utilizado la metodología ágil Scrum \cite{wiki:scrum}.
Los pasos que se han seguido con esta metodología son:
\begin{itemize}
\item Se define un tiempo, en nuestro caso entre dos semanas y un mes, llamado sprint. Al comienzo de cada intervalo se decidieron las tareas a realizar y el tiempo invertido en cada tarea.
\item Al final de cada sprint se tuvo otra reunión con el tutor (en un principio de forma presencial en la Universidad de Burgos y más adelante a través de Skype debido a mi traslado a Málaga) para poner en común el trabajo realizado, los problemas que surgieron y los siguientes pasos a seguir
\end{itemize}
El proyecto comenzó el 05/Abril/2019 y se dio por finalizado el 13/Febrero/2020. Cabe destacar que este tiempo no se ha dedicado íntegramente al proyecto, sino que se aprovechó también para acabar asignaturas pendientes y cursar un máster en Málaga.

\subsection{Sprint 1 (05/04/19 - 19/4/19)}
Primera reunión propiamente dicha del proyecto durante la que se acordó a que parte se dedicaría cada alumno. Esto es debido a que aunque sean trabajos independientes, el resultado final de este proyecto no es obra de una sola persona, sino que trabaje junto con mi compañero Jesús Manuel Calvo Ruiz de Temiño.

Se decidió que yo realizaría el trabajo del servidor mientras que mi compañero trabajaría en el cliente.

Las tareas acordadas fueron las siguientes:
\begin{itemize}
\item Documentarse sobre la metodología ágil.
\item Documentarse sobre glassfish\cite{wiki:glassfish} y realizar pequeños proyectos de prueba.
\end{itemize}

\subsection{Sprint 2 (19/4/19 - 29/04/19)}
Se propusieron las siguientes tareas:
\begin{itemize}
\item Realizar la conexión entre cliente y servidor usando las versiones del anterior proyecto.
\item Documentarse sobre bases de datos geoespaciales.
\item Documentarse sobre el formato de los datos de las ventanas de tiempo de los establecimientos en la base de datos OSM\cite{wiki:osm}
\item Obtener bases de datos geoespaciales sobre diferentes ciudades y hacer un estudio de los datos obtenidos para apreciar la calidad de los mismos y aprender a acceder a las diferentes etiquetas de los mismos.
\end{itemize}

\subsection{Sprint 3 29/04/19 - 13/05/19}
se hablo de los algoritmos
\subsection{Sprint 4 ( - 10/6/19)}

\section{Estudio de viabilidad}

\subsection{Viabilidad económica}

\subsection{Viabilidad legal}


