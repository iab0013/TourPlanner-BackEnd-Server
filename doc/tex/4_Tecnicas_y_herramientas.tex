\capitulo{4}{Técnicas y herramientas}

%Esta parte de la memoria tiene como objetivo presentar las técnicas metodológicas y las herramientas de desarrollo que se han utilizado para llevar a cabo el proyecto. Si se han estudiado diferentes alternativas de metodologías, herramientas, bibliotecas se puede hacer un resumen de los aspectos más destacados de cada alternativa, incluyendo comparativas entre las distintas opciones y una justificación de las elecciones realizadas. 
%No se pretende que este apartado se convierta en un capítulo de un libro dedicado a cada una de las alternativas, sino comentar los aspectos más destacados de cada opción, con un repaso somero a los fundamentos esenciales y referencias bibliográficas para que el lector pueda ampliar su conocimiento sobre el tema.
En este capítulo se analizarán brevemente las herramientas y técnicas utilizadas en la realización del proyecto.

\section{PostgreSQL}
% Sección dedicada a PostgreSQL.
PostgreSQL \cite{wiki:postgresql} es un sistema de gestión de bases de datos relacional orientado a objetos y de código abierto.
Se ha utilizado junto a diversas extensiones para adaptarlo a las necesidades del proyecto.

\section{PostGIS}
%Sección dedicada a PostGIS.
PostGIS (2) es una extensión de bases de datos espaciales para la base de datos relacional PostgreSQL. Agrega soporte para objetos geográficos permitiendo que las consultas de ubicación se ejecuten en SQL.
Un ejemplo de su uso en este proyecto es la obtención de diferentes puntos de interés con sus respectivos horarios a partir de unas coordenadas.